% Chapter Template

\chapter{Literature Review} % Main chapter title

\label{Chapter2} % Change X to a consecutive number; for referencing this chapter elsewhere, use \ref{ChapterX}

%----------------------------------------------------------------------------------------
%	SECTION 1   % TARGET 2000 WORDS IN THIS CHAPTER
%----------------------------------------------------------------------------------------

\section{Overview}  
The aim of this literature review is to highlight the importance of Universal Design and digital sovereignty in any design process as well as some insight into how this benefits all users. 
This is all linked back to the MEGAphone device, specifically, the work carried out by the author. %%possibly change, revise

%----------------------------------------------------------------------------------------
%	SECTION 2
%----------------------------------------------------------------------------------------

\section{History of Universal Design}
%% start with origin of UD
%% COVID, relevant to sovereignty, manufacturing
Universal Design (UD) is a concept that was first coined by the founder of 'The Center for Universal Design', Dr Ronald Mace at NC State University in the United States\cite{ronald}.
This concept was developed as there was a clear need for inclusion in product design due to the sizable number of individuals with a disability or 'of old age' in the United States. %%find census on aus as well
In a paper dubbed, 'The Universal Design File'\cite{universalfile}, Story and others noted that the average lifespan is far longer today than the beginning of the 20th century, due to better healthcare among other reasons.
They highlighted that, due to this factor as well as the previous two world wars has led to the United States having approximately 20 percent of their population living with some sort of disability.
This demographic is actually quite common, and so designing products with that in mind is more important than ever, and will only continue to gain relevance in the years to come.

With the introduction of the concept of Universal Design, new rules in which to govern this movement were established.
Story introduces the seven design principles \cite{sevenprinciples} intended for all design disciplines irrespective of whether they work in computer science or architecture.
It is important to highlight that Story's design principles should apply not only to physical products but also software, which is extremely relevant in an excellingly software oriented world.

One idea that is often confused with this concept is that the term, 'accessibility' is synonymous with 'disability' and this is not the case.
UD is a design method that was created to level the playing field by designing to reach the greater population, all ages and all abilities.
In a paper by Bringolf \cite{accessible}, she explains that as UD is still a relatively new concept, even today, it has generally been understood as designing for those with disabilities.
In Australia, a country with disability descrimination legislation (to protect people with disabilities), Bringolf notes that designers with this mindset will often design for this out of fear of litigation.
Bringolf observed that this creates an unhealthy approach to 'designing for all' which is what UD was originally created for. %% revise this, not sure it makes sense
This is why Bringolf wants to fight for the concept that UD is for everyone, and by extension this author also wishes to reiterate that idea looking at the development of this project deliverable. %%too meta?

%----------------------------------------------------------------------------------------
%	SECTION 3
%----------------------------------------------------------------------------------------

\section{Logistics of COVID-19}
%% discuss issues with supply in Australia due to COVID, need independence from imports
//

%----------------------------------------------------------------------------------------
%	SECTION 4
%----------------------------------------------------------------------------------------

\section{Mobile Device Accessibility}
%% talk about evolution of accessibility in mobile devices and discuss the problem with regard to the lack of accessibility supported
%% talk about how some mobile devices are being designed to include accessibility features and relate this all to sovereignty
Accessibility in mobile devices has in many ways been a part of the recipe from its introduction with features in a modern context such as vibration motors for alerts or text-to-speech for typing or reading content aloud.
However, as time has progressed, accessibility has slowly faded to the background to the point that modern mobile devices, commonly referred to as smartphones, have accessibility features that are far inferior to what they should be. %%does this flow well from first sentence?
A study by Law \cite{cellphone}, points out that there are a number of issues in regard to the lack of UD within the ICT industry. %%NEED ABREVIATION LIST
He discusses a key note at the time of writing in that very few mobile devices have 'out of the box' functionality to provide users with vision impairment a platform in which to interact with the device. %%relate to present day

Accessibility has since improved in the software aspect with smartphone manufacturers now implimenting features that allow users to navigate the device with low vision or blindness.
Both Apple and Android products now have features that allow complete blind interaction with the device in the form of Apple VoiceOver \cite{iphone} and Android's equivilent of TalkBack \cite{android} (view the following sections).

%-----------------------------------
%	SUBSECTION 1
%-----------------------------------

\subsection{Apple iPhone}
%% TALK ABOUT ACCESSIBLE DEVICES IN THESE SECTIONS
//

%-----------------------------------
%	SUBSECTION 2
%-----------------------------------

\subsection{Android}
//

% %----------------------------------------------------------------------------------------
% %	SECTION 5
% %----------------------------------------------------------------------------------------

% \section{Software Accessibility}
% //

% %-----------------------------------
% %	SUBSECTION 1
% %-----------------------------------

% \subsection{iOS}
% //

% %-----------------------------------
% %	SUBSECTION 2
% %-----------------------------------

% \subsection{Android}
% //

% %-----------------------------------
% %	SUBSECTION 3
% %-----------------------------------

% \subsection{Windows 10}
% Windows 10 is an operating system developed by Microsoft for the x86 architecture released in July 2015.

%----------------------------------------------------------------------------------------
%	SECTION 6
%----------------------------------------------------------------------------------------

\section{Repairability}
//

%----------------------------------------------------------------------------------------
%	SECTION 7
%----------------------------------------------------------------------------------------

\section{Summary}
//