% Chapter Template

\chapter{Literature Review} % Main chapter title

\label{Chapter2} % Change X to a consecutive number; for referencing this chapter elsewhere, use \ref{ChapterX}

%----------------------------------------------------------------------------------------
%	SECTION 1   % TARGET 2500 WORDS IN THIS CHAPTER
%----------------------------------------------------------------------------------------

\section{Overview} %% REVISE THESE OVERVIEWS, MAKE IT COVER STRICTLY CONTENT IN CHAPTER

This literature review will highlight the importance of Universal Design in any design process, along with details into that process and some insight into how this benefits all users.
While the focus of the project deliverable is moreso on Universal Design, the author wishes to highlight that there is more that designers could be doing to support the Digital Sovereignty movement as well, specifically the 'Right to Repair' concept, which is why this literature review also provides some insight into that aspect.
% This is all linked back to the MEGAphone device, specifically, the work carried out by the author. %%possibly change, revise

%----------------------------------------------------------------------------------------
%	SECTION 2
%----------------------------------------------------------------------------------------

\section{History of Universal Design}
%% start with origin of UD
%% COVID, relevant to sovereignty, manufacturing
Universal Design (UD) is a concept that was first coined by the founder of 'The Center for Universal Design', Dr Ronald Mace at North Carolina State University in the United States\cite{ronald}.
This concept was developed as there was a clear need for inclusion in product design due to the sizable number of individuals with a disability or 'of old age' in the United States.
In Australia, where this project is based, there are currently more than 4 million people with some form of disability \cite{ausstats} as of 2019. 
In a paper dubbed, 'The Universal Design File'\cite{universalfile}, Story and others noted that the average lifespan is far longer today than the beginning of the 20th century, due to better healthcare among other reasons.
They highlighted that, due to this factor as well as the previous two world wars has led to the United States having approximately 20 percent of their population living with some sort of disability, quite similiar to the Australian population.
This demographic is more common than people might think and so designing products with that in mind is now more important than ever before.

The introduction of the concept of Universal Design naturally led to new rules in which to govern any design process. %%reword this
Story introduces the seven design principles \cite{sevenprinciples} intended for all design disciplines irrespective of whether they work in computer science or architecture.
It is important to highlight that Story's design principles should apply not only to physical products but also software, which is extremely relevant in an excellingly software oriented world.
While not all principles will necessarily apply to every application, there is certainly more than one principle that will be relevant regardless of the field.
In the following figure, each of seven design principles developed by Story is listed followed by a brief summary of each principle.

\begin{center}
    \begin{tabular}{ |c|c| } 
     \hline
     Principle One & Equitable Use \\
     \hline
     Principle Two & Flexibility in Use \\ 
     \hline
     Principle Three & Simple and Intuitive Use \\ 
     \hline
     Principle Four & Perceptible Information \\
     \hline 
     Principle Five & Tolerance for Error \\ 
     \hline
     Principle Six & Low Physical Effort \\ 
     \hline
     Principle Seven & Size and Space for Approach and Use \\
     \hline
    \end{tabular}
\end{center}

\begin{figure}
    \caption{A brief description and analysis of the seven design principles by Story\cite{sevenprinciples}} %%fix this, NOT A FIGURE, is a table
    \label{fig:DesignPrinciples}
\end{figure}

The first principle states that the design should be appealing to all users, meaning that it should be accessible to everyone regardless of their personal situation.
The second principle explains that the design should be adaptable, meaning that someone who perhaps has poor dexterity or poor vision is able to successfully interact with the product and otherwise, an experienced user should be able to interact without feeling frustrated. %%maybe revise
Principle three ensures that unnecessary complexity is eliminated which includes making the device appear consistent with user expectations along with a clear distinction of the importance of information, meaning that it is presented logically and in order.
The forth principle focuses on legibility of essential information by ensuring that elements of the design are unique and easy to distinguish including by those with limited sensory abilities.
Principle five looks at how potential hazards in a design can be isolated and how the user can be alerted to any potentially dangerous elements of a design.
The sixth principle covers user effort, in that the experience should be comfortable and actions that require a lot of interation causes minimal fatigue.
Principle seven states that the size and layout of design elements should be appropriately thought out so that interaction with various features is comfortable for any user regardless of whether they are in a standing or seated position.
There is a common theme in these principles in that designers should develop their products so that they are adaptable to someone who lacks certain sensory or physical abilities however, this does not mean 'specialised' design, but rather 'inclusive' design.
%% Perhaps comment on these principles?

One idea that is often confused with the UD concept is that the term, 'accessibility' is synonymous with 'disability' and this is not the case.
UD is a design method that was created to level the playing field by designing to reach the greater population, all ages and all abilities.
In a paper by Bringolf\cite{accessible}, she explains that as UD is still a relatively new concept, even today, it has generally been understood as designing for those with disabilities.
In Australia, a country with disability descrimination legislation (to protect people with disabilities), Bringolf notes that designers with this mindset will often design for this out of fear of litigation.
Bringolf observed that this creates an unhealthy approach to 'designing for all' which is what UD was originally created for. %% revise this, not sure it makes sense
This is why Bringolf wants to fight for the concept that UD is for everyone, and by extension this author also wishes to reiterate that idea looking at the development of this project deliverable. %%too meta?
It is worth noting that while certain assistive technologies will always be required for those with significant needs, Story, the creator of the seven principles, explains that Universal Design is about making mainstream products as accessible as possible\cite{sevenprinciples}.

%----------------------------------------------------------------------------------------
%	SECTION 3
%----------------------------------------------------------------------------------------

\section{Mobile Device Accessibility}
%% talk about evolution of accessibility in mobile devices and discuss the problem with regard to the lack of accessibility supported
%% talk about how some mobile devices are being designed to include accessibility features and relate this all to sovereignty
Accessibility in mobile devices has in many ways been a part of the recipe from its introduction with features in a modern context such as vibration motors for alerts or text-to-speech for typing or reading content aloud.
However, as time has progressed, accessibility has slowly faded to the background to the point that modern mobile devices, commonly referred to as smartphones, have accessibility features that are far inferior to what they should be. %%does this flow well from first sentence?
A study by Law\cite{cellphone}, points out that there are a number of issues in regard to the lack of UD within the ICT industry. %%NEED ABREVIATION LIST
He discusses a key note at the time of writing in that very few mobile devices have 'out of the box' functionality to provide users with vision impairment a platform in which to interact with the device. %%relate to present day

The Disability Discrimination Act of 1992\cite{dda1992} prohibits the mistreatment of those with a disability in all areas of life.
While this act protects those individuals from discrimination in situations such as seeking employment, sale of goods, or buying a house, there is no specific clause that covers mobile device accessibility or software in general.
Despite this, accessibility has since improved in the software aspect with smartphone manufacturers now implimenting features that allow users to navigate the device with low vision or blindness.
Both Apple and Android smartphone products have features that allow complete blind interaction with the device in the form of Apple VoiceOver \cite{iphone} and Android's equivilent of TalkBack \cite{android} (view section 3.1 and 3.2).
However, while these features should be praised for doing good on the part of making their devices more accessible to vision impaired, or deaf users, these devices are built with aesthestics in mind and with accessiblity as an afterthought.

Universal Design is a concept that is only successful when accessibility is a part of the design philosophy from the beginning\cite{incldesign}.
In a paper by Newell and others\cite{incldesign}, they found that designers who adopted the Universal Design concept would often follow traditional methods of design and then go back and investigate how they might adapt the design to support a larger audience.
The problem with this was that they would end up with products that would appear to be accessible, however in reality, were actually quite difficult for people with disablities and the elderly.
In an earlier paper by Newell and others, they highlight the rapidly increasing demographic of people with disabilities and elderly people interacting with computers shows that accessible interfaces are actually of very great importance\cite{computerinterface}.
This is certainly much more true today, with computers in the form of smartphones which have become a modern day tool, present in everyday life, for very a very large demographic of people.

%-----------------------------------
%	SUBSECTION 1
%-----------------------------------

\subsection{VoiceOver}
%% TALK ABOUT ACCESSIBLE DEVICES IN THESE SECTIONS
Apple's VoiceOver accessibility software running on iOS provides users with a gesture-based, screen reader in which to make interaction with the interface easier\cite{iphone}.
VoiceOver provides read aloud features for all screen elements on screen as well as those typed by the user.
Other standard features include giving users control over the colour and contrast of the display as well as the ability to change the scale of elements such as the font, or the screen in its entirety.
One facinating feature of the VoiceOver software is that it provides users with a 6 and 8 dot braille keyboard, which has seen to be a very effective alternative to a QWERTY keyboard [source]. %%cite that youtube video of blind person typing

One aspect that perhaps sets the device back is that while it has full support for built in apps, not all third-party apps are supported.
However, this is something that would fall on third-party app developers to impliment and Apple does do their diligence to provide developers with resources for this purpose\cite{iphonedev}.
Overall, VoiceOver does well to make the most with a platform that was not designed with UD in focus from the start.

%-----------------------------------
%	SUBSECTION 2
%-----------------------------------

\subsection{TalkBack}
Android's TalkBack works very similiarly to their Apple counterpart in that they allow the user to explore by touch by dragging their finger around the screen to recieve text-to-speech announcements\cite{android}.
Talkback allows users to read all elements on the screen continously and use gestures in order to navigate the device. %%need to cite these pages separately most likely
TalkBack also provides the aforementioned on-screen braille keyboard support with the 6 and 8 dot options, alternatively referred to as 'grade 1' and 'grade 2' braille depending on the experience of the user.

%----------------------------------------------------------------------------------------
%	SECTION 4
%----------------------------------------------------------------------------------------

\section{Hardware Accessibility}

Hardware accessibility in the realm of universal design is an area that will see much focus in this thesis. %%again too meta? double check
The following section reviews advancements made in the video game industry in regard to designing for accessibility.

%-----------------------------------
%	SUBSECTION 1
%-----------------------------------

\subsection{Accessible Game Controller}

Microsoft's Xbox Adaptive Controller\cite{adaptive}, is a gaming controller that was designed in collaboration with a number of US associations and gamers with a disability, with accessibility in full focus.
This controller is designed with a large array of nineteen 3.5mm jacks and two USB 2.0 ports for external inputs\cite{adaptive} to allow a large degree of customisation for users with disabilities, so that they can plug in any digital switch input to suit their needs.
It is also in the affordable price range at 130 Australian dollars\cite{accessiblecontroller} at the time of writing, compared to a 'standard' wireless controller which retails at 90 Australian dollars\cite{standardcontroller} Microsoft online store.

The Xbox Adaptive controller is the first commercially available truely accessible controller however, is not the first instance of this, as an earlier prototype revealed in 2011, called the Adroit Switchblade\cite{ablegamer}, in many ways set the groundwork in a partnership that lead to Microsoft's own adaptation.
The 
In an article interviewing Bryce Johnson, lead inclusive designer for Microsoft, Johnson states that "our standard controller has been optimised over the years around a primary use case – two thumbs and two index fingers”\cite{disabilitygaming}, a statement that he goes on to say has placed a barrier against playing for some people.
It is also evident that Microsoft %%FIX THIS

% An article by Grammenos and others 

%----------------------------------------------------------------------------------------
%	SECTION 5
%----------------------------------------------------------------------------------------

\section{Right to Repair}

The 'Right to Repair' mantra falls under the Digital Sovereignty movement, where focus is on how products can be designed to be transparent to the user and fully repairable by them [SOURCE?].
This is a concept that has recently seen more attention with electronics in the light of a trend where manufacturers are making it increasingly difficult for individuals to repair their own devices.
Electronic products such as smartphones are being designed with a limited lifespan under a concept that a paper by Bulow\cite{obsolescence} refers to as 'Planned Obsolescence', to encourage repeat purchases from customers in order to maximise profits.
Bulow goes on to also state that monopolists have an incentive to produce products with less durability as this takes away from the cost of developing a higher quality product.

On top of this, an article from Waldman shows that monopolists also have a high incentive to introduce a new electronic device with new sylistic changes, which in turn makes the preceeding device obsolete\cite{obsolescence2}.
Waldman at the time, noted that technology company IBM, had changed their operating system in new computers which would no longer be compatible with older products, once again, an example of planned obsolescence.
Apple's iPhone, among other competitors, is a clear example of planned obsolescence at work today, with the consumer culture very much in favour of having with the 'latest and greatest' smartphone in their hand.
While it is important to note that consumer electronics are constantly innovating and therefore evolving, the distinction in this case is that manufacturers are not making their products 'upgradeable', which contributes to the shorter lifespan of the product.

As digitally sovereign devices are becoming increasingly rare due to a number of factors such as increasing complexity and protection of trade secrets, the MEGAphone device is an initative that aims to reclaim this dying concept. %%edit this, revise
The appeal of the MEGAphone device is that it is intended as a platform in which users can modify and repair as they desire, and is therefore completely against planned obsolescence.
In a CCC (Chaos Communication Camp) talk, Gardner-Stephen explains that their approach is to "start obsolete" so that they can't "get obsolete" and reduce the complexity of features to make it as accessible to individuals as possible\cite{mobilehistory}.
While this can be considered a niche approach, as true Digital Sovereignty is often sacrificed for a number of reasons such as the high complexity of modern operating systems (OS), Gardner-Stephen goes on to state that the Commodore 64 OS among others proved that simple architecture can still achieve a great deal.

%----------------------------------------------------------------------------------------
%	SECTION 6
%----------------------------------------------------------------------------------------

\section{Effects of COVID-19}
%% discuss issues with supply in Australia due to COVID, need independence from imports
The recent global outbreak of the COVID-19 virus has threatened the Digital Sovereignty movement and otherwise the 'Right to Repair' due to Australia's international trade dependence, particularly with China [SOURCE?].
%% PROBABLY ADDRESS THIS AFTER EVERYTHING ELSE

%----------------------------------------------------------------------------------------
%	SECTION 7
%----------------------------------------------------------------------------------------

\section{Summary}
//