% Chapter Template

\chapter{Introduction}\label{chapter:firstchapter} % Main chapter title

\label{Chapter1} % Change X to a consecutive number; for referencing this chapter elsewhere, use \ref{ChapterX}

%----------------------------------------------------------------------------------------
%	SECTION 1
%----------------------------------------------------------------------------------------

\section{Introduction}\label{sec:firstsection}

%% give some background on importance and relevance of universal design
%% explain issue that needs to be addressed using this concept
%% mention this is a new project

To increase the accessibility of the MEGAphone mobile device, this project aims to create a universally designed case for the current revision of the project.
The intention is to give enhanced telephony access to everyone, regardless of physical impairments, which is why the concept of Universal Design is vital to the success of this project. 
The term ‘Universal Design’ is a concept first coined by Ron Mace where “the design of products, environments, programmes and services to be usable by all people, to the greatest extent possible, without the need for adaptation or specialized design” \cite{nda}. 
Universal Design is an iterative process and therefore must be at the centre of the design philosophy from the start \cite{incldesign} which is why the MEGAphone chassis deliverable of this project is entirely new, based on existing PCB constraints.
Existing design principles developed at the Center for Universal Design at North Carolina State University \cite{sevenprinciples} will be adopted to help guide this project in the right direction.

This project will require Autodesk 3D CAD modeling software (or equivalent) in order to successfully translate concept sketches into a real tangible design. 
Accessible design practices will play a central role in the design process as the success of the final product depends on its ability to provide accessibility to all (regardless of age or disability). 
This project does have existing constraints, in that it is adapted around an existing PCB, however this design will take every opportunity to uphold the aforementioned design philosophy, as well as a possible redesign following these design practices.
Additionally, this project will look into how UD can incorporate digital sovereignty so that users can not only have an accessible interface, but one that can be easily repaired, modified and controlled by them.

%\begin{figure}
%\begin{centering}
%\includegraphics[width=10cm,height=10cm,keepaspectratio]{Figures/dont-panic-e1534046233310.jpg}
%\caption{The Hitch Hiker's Guide To The Galaxy (not to be confused with \cite{Reference1}. Image Credit David Strine (License: CC0)}
%\label{fig:ThisFig}
%\end{centering}
%\end{figure}

%----------------------------------------------------------------------------------------
%	SECTION 2
%----------------------------------------------------------------------------------------
\section{Background on the MEGAphone}

%% how was this project formed, and who works on it
The ‘MEGAphone’ is a digitally sovereign mobile device created by Dr Paul Gardner-Stephen with the intent to promote digital sovereignty \cite{mega65}.
This project began when a trend was noticed in modern consumer electronics, a trend that still exists today, in that companies design their products with ‘planned obsolescence’ in mind. 
Gardner-Stephen believes that users should have the ability to repair or modify any aspect of their device without difficulty \cite{mobilehistory}.

%% explain the purpose of FPGA, and security features
The MEGAphone incorporates the use of a field programmable gate-array (FPGA), a hardware programmable CPU designed to mimic the unreleased Commodore 65. 
This removes the need for emulation as well as providing modern performance advancements partially due to eliminating the need for costly emulation.

% (Section \ref{sec:firstsection}).

%----------------------------------------------------------------------------------------
%	SECTION 3
%----------------------------------------------------------------------------------------

\section{Project Objectives}

The main objective of this project, as outlined in the thesis is to design a prototype case for the MEGAphone that follows the seven Universal Design (UD) principles established at the Universal Design branch of the North Carolina State University \cite{sevenprinciples}. 
This thesis will also aim to document the process of design as well as any modifications to the design and why these changes were made. 
In terms of the research component, this thesis aims to address the relevance of UD in the design process as well as identifying how device sovereignty can be supported during this process. 
The desired output of this project as is documented will be a physical prototype of the MEGAphone case as well as software-side accessibility features for the MEGA65 operating system. 
A physical prototype of this project will allow for demonstration of the UD features.
 
%----------------------------------------------------------------------------------------
%	SECTION 4
%----------------------------------------------------------------------------------------

\section{Research Questions}

This thesis aims to answer multiple research questions:

\begin{itemize} 
    \item What is the relationship between digital sovereignty and universal design?
    \item Why is Universal Design important in the design of products today?
    \item Why is digital sovereignty important in the design of electronic products today?
    \item How can the MEGAphone device case be designed to support those with disabilities?
    \item What are the effects of COVID-19 on digital sovereignty and how does that affect the MEGAphone?
    \end{itemize}

%----------------------------------------------------------------------------------------
%	SECTION 5
%----------------------------------------------------------------------------------------

\section{Scope of Project}

This project includes development of computer-aided design chassis prototype, PCB design and manufacturing for device switches and adapters and accessibility software, written in VHDL for the MEGA65 operating system.

%----------------------------------------------------------------------------------------
%	SECTION 6
%----------------------------------------------------------------------------------------

\section{Layout of Thesis}

Chapter 1 addresses the aims of the project, provides some context on the MEGAphone and lists the questions that this thesis aims to answer. 
Chapter 2 investigates how Universal Design can help to overcome many issues in product design that people with disabilities face without the need for specialized equipment. 
Additionally, the lack of digital sovereignty in electronics devices is addressed as well as why it is imperative that it be preserved.
Chapter 3 focuses on the methods and tools used to plan, design and manufacture this project.
Chapter 4 discusses the design process and the incremental changes made to the design toward a definitive prototype, supplemented by a discussion in regard to the outcome of the project. 
Chapter 5 provides concluding remarks as well as a final evaluation of each revision of the case and the accompanying software.