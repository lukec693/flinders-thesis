% Chapter Template

\chapter{Introduction}\label{chapter:firstchapter} % Main chapter title

\label{Chapter1} % Change X to a consecutive number; for referencing this chapter elsewhere, use \ref{ChapterX}

%----------------------------------------------------------------------------------------
%	SECTION 1
%----------------------------------------------------------------------------------------

\section{Introduction}\label{sec:firstsection}

% It is a good idea to have each sentence on a separate line, so that if you get feedback or changes from someone else
% the diffs will be much easier to manage
//give some background on importance and relevance of universal design
//explain issue that needs to be addressed using this concept
// mention this is a new project,

To increase the accessibility of the MEGAphone mobile device, this project aims to create a universally designed case for the current revision of the project.
The intention is to give enhanced telephony access to everyone, regardless of physical impairments. 
The term ‘Universal Design’ is a concept first coined by Ron Mace where “the design of products, environments, programmes and services to be usable by all people, to the greatest extent possible, without the need for adaptation or specialized design” (National Disability Authority, nd). 
Existing design principles developed by the Center for Universal Design at North Carolina State University (Story, 1998) will be adopted to help guide this project in the right direction.

    This is a new project, which will require Autodesk 3D CAD modeling software (or equivalent) in order to successfully translate concept sketches into real tangible designs. 
Inclusive design practices will play a central role in the design process as the success of the final product depends on its ability to provide accessibility to all (regardless of age or disability). 
Universal Design is an iterative process and therefore must be at the centre of the design philosophy from the start (Newel et al., 2011). 
Additionally, this project will look into how UD can incorporate digital sovereignty so that users can not only have an accessible interface, but one that can be easily repaired, modified and controlled by them.

\begin{figure}
\begin{centering}
\includegraphics[width=10cm,height=10cm,keepaspectratio]{Figures/dont-panic-e1534046233310.jpg}
\caption{The Hitch Hiker's Guide To The Galaxy (not to be confused with \cite{Reference1}. Image Credit David Strine (License: CC0)}
\label{fig:ThisFig}
\end{centering}
\end{figure}

%-----------------------------------
%	SUBSECTION 1
%-----------------------------------
\subsection{Subsection 1}

Nunc posuere quam at lectus tristique eu ultrices augue venenatis (Chapter \ref{chapter:firstchapter}).
Vestibulum ante ipsum primis in faucibus orci luctus et ultrices posuere cubilia Curae; Aliquam erat volutpat.
Vivamus sodales tortor eget quam adipiscing in vulputate ante ullamcorper.
Sed eros ante, lacinia et sollicitudin et, aliquam sit amet augue.
In hac habitasse platea dictumst (Section \ref{sec:firstsection}).

%-----------------------------------
%	SUBSECTION 2
%-----------------------------------

\subsection{Subsection 2}
Morbi rutrum odio eget arcu adipiscing sodales.
Aenean et purus a est pulvinar pellentesque.
 Cras in elit neque, quis varius elit.
 Phasellus fringilla, nibh eu tempus venenatis, dolor elit posuere quam, quis adipiscing urna leo nec orci.
 Sed nec nulla auctor odio aliquet consequat.
 Ut nec nulla in ante ullamcorper aliquam at sed dolor.
 Phasellus fermentum magna in augue gravida cursus.
 Cras sed pretium lorem.
 Pellentesque eget ornare odio.
 Proin accumsan, massa viverra cursus pharetra, ipsum nisi lobortis velit, a malesuada dolor lorem eu neque.

%----------------------------------------------------------------------------------------
%	SECTION 2
%----------------------------------------------------------------------------------------

\section{Main Section 2}

Sed ullamcorper quam eu nisl interdum at interdum enim egestas.
 Aliquam placerat justo sed lectus lobortis ut porta nisl porttitor.
 Vestibulum mi dolor, lacinia molestie gravida at, tempus vitae ligula.
 Donec eget quam sapien, in viverra eros.
 Donec pellentesque justo a massa fringilla non vestibulum metus vestibulum.
 Vestibulum in orci quis felis tempor lacinia.
 Vivamus ornare ultrices facilisis.
 Ut hendrerit volutpat vulputate.
 Morbi condimentum venenatis augue, id porta ipsum vulputate in.
 Curabitur luctus tempus justo.
 Vestibulum risus lectus, adipiscing nec condimentum quis, condimentum nec nisl.
 Aliquam dictum sagittis velit sed iaculis.
 Morbi tristique augue sit amet nulla pulvinar id facilisis ligula mollis.
 Nam elit libero, tincidunt ut aliquam at, molestie in quam.
 Aenean rhoncus vehicula hendrerit.
