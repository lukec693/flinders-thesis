% Chapter Template

\chapter{Introduction}\label{chapter:firstchapter} % Main chapter title

\label{Chapter1} % Change X to a consecutive number; for referencing this chapter elsewhere, use \ref{ChapterX}

%----------------------------------------------------------------------------------------
%	SECTION 1   % TARGET 1500 WORDS IN THIS CHAPTER
%----------------------------------------------------------------------------------------

\section{An Overview}\label{sec:firstsection}

%% give some background on importance and relevance of universal design
%% explain issue that needs to be addressed using this concept
%% mention this is a new project
%% frame this all around a gap statement
In today's day and age where humans are living longer than ever before, the very young, elderly and people with a disability is a demographic that is constantly growing as populations increase.
In Australia, where this project is based, the current population of people with a disability represents an estimation of 1 in 6 people\cite{ausstats}, a group that is far more common than many may realise.
This has led to the concept of Universal Design (UD), established North Carolina State University by founder Ron Mace with the intention of providing accessible design to all people, regardless of age or disability.

This project will use the concept of UD to develop an accessible platform for the MEGAphone mobile device, a secure mobile device developed by Dr Paul Gardner-Stephen and the MEGA65 team based in Germany.
The intention of this project is to give enhanced telephony access to everyone, regardless of physical impairments, which is why the concept of Universal Design (UD) is vital to the success of this project. 
The term ‘Universal Design’ is a concept first coined by Ron Mace where “the design of products, environments, programmes and services to be usable by all people, to the greatest extent possible, without the need for adaptation or specialized design” \cite{nda}. 
Universal Design is an iterative process and therefore must be at the centre of the design philosophy from the start \cite{incldesign} which is why the MEGAphone chassis deliverable of this project is entirely new, based on existing PCB constraints.
Seven existing design principles developed at the Center for Universal Design at North Carolina State University \cite{sevenprinciples} will be adopted to help guide this project in the right direction.
The aim of this project will be to create an accessible case for the current, second revision of the project, supported by these seven principles.

This project will require Autodesk 3D CAD modeling software (or equivalent) in order to successfully translate concept sketches into a real tangible design. 
Accessible design practices will play a central role in the design process as the success of the final product depends on its ability to provide accessibility to all (regardless of age or disability). 
This project does have existing constraints, in that it is adapted around an existing PCB, however this design will take every opportunity to uphold the aforementioned design philosophy and include a proposed redesign following these design practices.
Additionally, this project will look into how UD can incorporate Digital Sovereignty so that users can not only have an accessible interface, but one that can be easily repaired, modified and controlled by them.

%\begin{figure}
%\begin{centering}
%\includegraphics[width=10cm,height=10cm,keepaspectratio]{Figures/dont-panic-e1534046233310.jpg}
%\caption{The Hitch Hiker's Guide To The Galaxy (not to be confused with \cite{Reference1}. Image Credit David Strine (License: CC0)}
%\label{fig:ThisFig}
%\end{centering}
%\end{figure}

%----------------------------------------------------------------------------------------
%	SECTION 2
%----------------------------------------------------------------------------------------
\section{Background on the MEGAphone}

%% how was this project formed, and who works on it
The ‘MEGAphone’ is a digitally sovereign mobile device created by Dr Paul Gardner-Stephen and the MEGA65 team in Germany with the intent to promote Digital Sovereignty \cite{mega65}.
This project began when a trend was noticed in modern consumer electronics, a trend that still exists today, in that companies design their products with ‘planned obsolescence’ in mind. 
Gardner-Stephen believes that users should have the ability to repair or modify any aspect of their device without difficulty \cite{mobilehistory}.

%% explain the purpose of FPGA, and security features
The MEGAphone incorporates the use of a field programmable gate-array (FPGA), a hardware programmable CPU designed to mimic the unreleased Commodore 65.
This eliminates the costly alternative of emulation and provides the option for modern performance advancements in aspects such as storage or boot time.
One of the marketable assets of this project is that due to the simplicity of the system, the device is very secure.
This coupled with the addition of hardware power switches, gives users the piece of mind that insecure wireless modules cease to operate as desired by the user.

% (Section \ref{sec:firstsection}).

%----------------------------------------------------------------------------------------
%	SECTION 3
%----------------------------------------------------------------------------------------

\section{Project Objectives}

The main objective of this project is to design a prototype case for the MEGAphone that follows the seven principles of Universal Design (UD), established at the Center for Universal Design, North Carolina State University \cite{sevenprinciples}. 
Other deliverables will be in support of this case and will include, PCBs to assist with add-on features and hardware-implimented accessiblity features in the FPGA.
This thesis will document the design process including any modifications to the designs, along with a justification as to why specific changes were made as opposed to alternative options.
In terms of the research component, this thesis aims to address the relevance of UD in the design process as well as identifying the importance of device sovereignty and how it can be supported during this process.
Specifically, this thesis will look into the history of UD, some examples of how certain features have been implimented in existing products, both hardware and software, and how the effects of the COVID-19 have negatively affected the ability for users to source their own parts for repairs under the umbrella of Digital Sovereignty.
The desired output of this project as documented will be a physical prototype of the MEGAphone case and supporting components as well as software-side accessibility features for the MEGA65 operating system.
Multiple physical prototypes of this project will allow for not only a demonstration of the UD features but also a demonstration of the engineering design iteration process.
 
%----------------------------------------------------------------------------------------
%	SECTION 4
%----------------------------------------------------------------------------------------

\section{Research Questions}

This thesis aims to answer multiple research questions:

\begin{itemize} 
    \item What is the relationship between Digital Sovereignty and Universal Design?
    \item Why is Universal Design important in the design of products today?
    \item Why is Digital Sovereignty important in the design of electronic products today?
    \item How can the seven design principles be used in the design of the MEGAphone case to support those with disabilities?
    \item How can the 'Right to Repair' mantra be used in support of the universally designed MEGAphone case?
    \item What are the effects of COVID-19 on Digital Sovereignty and how does that affect the MEGAphone? %%this perhaps too late in project to mention?
    \end{itemize}

%----------------------------------------------------------------------------------------
%	SECTION 5
%----------------------------------------------------------------------------------------

\section{Scope of Project}

This project will look at the history and importance of UD and Digital Sovereignty (specifically the 'Right to Repair') in the design of products today as well as the relationship between those two concepts.
This is all supported by a dive into the development of a unique computer-aided universally designed chassis prototype graded against the seven design principles\cite{sevenprinciples}, along with multiple PCB designs, PCB manufacturing and hardware-implimented accessibility 'software' written in VHDL for the MEGA65 operating system.
%% possible update, revise this at some point

%----------------------------------------------------------------------------------------
%	SECTION 6
%----------------------------------------------------------------------------------------

\section{Layout of Thesis}

Chapter 1 addresses the aims of the project, provides some context on the MEGAphone and lists the questions that this thesis aims to answer.
Chapter 2 investigates how Universal Design can help to overcome many issues in product design that people with disabilities face without the need for specialized equipment. 
Additionally, the lack of Digital Sovereignty in electronics devices is addressed as well as why it is imperative that it be preserved.
Chapter 3 focuses on the methods and tools used to plan, design and manufacture this project.
Chapter 4 discusses the design process and the incremental changes made to the design toward a definitive prototype, supplemented by a discussion in regard to the outcome of the project. 
Chapter 5 provides concluding remarks as well as a final evaluation of the case prototype, PCBs, hardware-implemented 'software' features and future work to be undertaken.

%%% ADDRESS THIS SECTION AGAIN AT THE END!!

% %----------------------------------------------------------------------------------------
% %	SECTION 7
% %----------------------------------------------------------------------------------------

% \section{Contribution to the Field}
% //