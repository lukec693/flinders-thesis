% Chapter Template

\chapter{Introduction}\label{chapter:firstchapter} % Main chapter title

\label{Chapter1} % Change X to a consecutive number; for referencing this chapter elsewhere, use \ref{ChapterX}

%----------------------------------------------------------------------------------------
%	SECTION 1   % TARGET 1500 WORDS IN THIS CHAPTER
%----------------------------------------------------------------------------------------

\section{An Overview}\label{sec:firstsection} %%motivation

%% give some background on importance and relevance of universal design
%% explain issue that needs to be addressed using this concept
%% mention this is a new project
%% frame this all around a gap statement
\subsection{Disabilities affect many people}
In today's day and age where humans are living longer than ever before, the very young, elderly and people with a disability is a demographic that is constantly growing as populations increase.
In Australia, where this project is based, the current population of people with a disability represents an estimation of 1 in 6 people \cite{ausstats}, a group that is far more common than many may realise.
This has indirectly led to the concept of Universal Design (UD), established at North Carolina State University by founder Ron Mace with the intention of providing accessible design to all people, regardless of age or disability \cite{ronald}.
The term ‘Universal Design’ is a concept first coined by Ron Mace where “the design of products, environments, programmes and services to be usable by all people, to the greatest extent possible, without the need for adaptation or specialized design” \cite{nda}. 

\subsection{Digital Sovereignty and Capability Maintenance of accessibility applications} %%look up and add capability maintenance
On a different yet related note, individuals are losing out on the right to repair their own electronic devices, as the everpresent concept of planned obsolescence continues to dominate the market due to the undeniable profitability of doing so \cite{obsolescence2}.
%A number of factors such as increased complexity of computer systems and the desire of companies to protect their intellectual property, mean that true Digital Sovereignty is unrealistic. %%%CONTRADICTORY
This problem threatens the Digital Sovereignty movement as it serves as the overarching concept of 'Right to Repair', which has seen an avenue for the MEGAphone device developed by Dr Paul Gardner-Stephen \cite{mobilehistory}, based on a project developed by the MEGA65 team \cite{mega65}, to bring life to the unreleased Commodore65 computer. %%revise
Gardner-Stephen's goal is to provide users with a secure mobile device, built on a far less complex and therefore more transparent computer system than modern alternatives, of which the right to repair, modify and maintain stays with the user.

\subsection{MEGAphone Digitally Sovereign smartphone concept}
This project aims to use the concept of UD to develop an accessible platform for the MEGAphone smartphone with the intention of giving enhanced telephony access to everyone, regardless of physical impairments, or age.
The concept of UD is vital to the success of this project, as it lays the groundwork to achieve a truely open, truely accessible platform for all users. 
This groundwork refers to the seven design principles, a set of design guidelines developed by the Center for Universal Design at North Carolina State University \cite{sevenprinciples}, with the purpose of bringing clarity to the development of an accessible design.
This will be achieved while staying true to the Digital Sovereignty movement by providing users with a platform that they can maintain, modify and repair with the bonus of being more simple and intuitive to use.

\subsection{Goals of this project}
The MEGAphone chassis deliverable (otherwise referred to as a 'case' or 'housing') is an entirely new project, based on existing PCB constraints, which is important to define as Universal Design is an iterative process and therefore must be at the centre of the design philosophy from the start \cite{incldesign}.
The aim of this project will be to create an entirely new accessible case for the current revision of the project, supported by the seven design principles under the 'umbrella' of UD.

This project will require 3D parametric computer aided design (CAD) modeling software in order to successfully translate concept sketches into a real tangible design.
Another requirement will be electronic CAD design software, to design the accompanying PCBs for this project as there will be a requirement to adapt to the existing constraints of the MEGAphone PCB.
The third requirement will be the use of Xilinx's Vivado FPGA programming software in order to design accessible user interface features.

Accessible design practices will play a central role in the design process as the success of the final product depends on its ability to provide accessibility to all (regardless of age or disability). 
As mentioned, this project does have existing constraints, in that it is adapted around an existing PCB, however this design will take every opportunity to uphold the aforementioned design philosophy and include a proposed redesign following these design practices.
Additionally, this project will look into how UD can incorporate Digital Sovereignty, more specifically, the 'Right to Repair' movement, so that users can not only have an accessible interface, but one that can be easily repaired, modified and controlled by them.

%% planned ob works against people with a disability, capability maintenance

%\begin{figure}
%\begin{centering}
%\includegraphics[width=10cm,height=10cm,keepaspectratio]{Figures/dont-panic-e1534046233310.jpg}
%\caption{The Hitch Hiker's Guide To The Galaxy (not to be confused with \cite{Reference1}. Image Credit David Strine (License: CC0)}
%\label{fig:ThisFig}
%\end{centering}
%\end{figure}

%----------------------------------------------------------------------------------------
%	SECTION 2
%----------------------------------------------------------------------------------------
\section{Background on the MEGAphone} %% now talking about this, because previous section

%% how was this project formed, and who works on it
The ‘MEGAphone’ is a digitally sovereign mobile device created by Dr Paul Gardner-Stephen and the MEGA65 team in Germany with the intent to promote Digital Sovereignty \cite{mega65}.
This project began when a trend was noticed in modern consumer electronics, a trend that still exists today, in that companies design their products with ‘planned obsolescence’ in mind. 
Gardner-Stephen believes that users should have the ability to repair or modify any aspect of their device without difficulty \cite{mobilehistory}.

%% explain the purpose of FPGA, and security features
The MEGAphone incorporates the use of a field programmable gate-array (FPGA), a hardware programmable CPU designed to mimic the unreleased Commodore 65.
This eliminates the costly alternative of emulation and provides the option for modern performance advancements in aspects such as storage or boot time.
One of the marketable assets of this project is that due to the simplicity of the system, the device is very secure.
This coupled with the addition of hardware power switches, gives users the piece of mind that insecure wireless modules cease to operate as desired by the user.
% (Section \ref{sec:firstsection}).

%----------------------------------------------------------------------------------------
%	SECTION 3
%----------------------------------------------------------------------------------------
\section{Project Objectives}

The main objective of this project is to design a prototype case for the MEGAphone that follows the seven principles of Universal Design (UD), established at the Center for Universal Design, North Carolina State University \cite{sevenprinciples}. 
Other deliverables will be in support of this case and will include, PCBs to assist with add-on features and hardware-implimented accessiblity features in the FPGA.
This thesis will document the design process including any modifications to the designs, along with a justification as to why specific changes were made as opposed to alternative options.
In terms of the research component, this thesis aims to address the relevance of UD in the design process as well as identifying the importance of device sovereignty and how it can be supported during this process.

Specifically, this thesis will look into the history of UD, some examples of how certain features have been implimented in existing products, both hardware and software, and how the effects of the COVID-19 have negatively affected the ability for users to source their own parts for repairs under the umbrella of Digital Sovereignty.
The desired output of this project as documented will be a physical prototype of the MEGAphone case and supporting components as well as software-side accessibility features for the MEGA65 operating system.
Multiple physical prototypes of this project will allow for not only a demonstration of the UD features but also a demonstration of the engineering design iteration process.
 
%----------------------------------------------------------------------------------------
%	SECTION 4
%----------------------------------------------------------------------------------------
\section{Research Questions}

This thesis aims to answer multiple research questions:

\begin{enumerate}
    \item What is the relationship between Digital Sovereignty and Universal Design? %% explain why
        \begin{enumerate}
        \item[-] This project should establish how UD and DS can be used to make a mobile device that not only satisfies both concepts but actively shows how they can be mutually beneficial.
        \end{enumerate} 
    \item Why is Universal Design important in the design of products today?
        \begin{enumerate}
        \item[-] Support the concept that UD should be used in any design project, as it benefits all users to a greater extent.
        \end{enumerate}
    \item Why is Digital Sovereignty important in the design of electronic products today?
        \begin{enumerate}
        \item[-] Show how DS is important to individuals with their electronics as it protects their right to a secure and open platform.
        \end{enumerate} 
    \item How can the seven design principles be used in the design of the MEGAphone case to support those with disabilities?
        \begin{enumerate}
        \item[-] Demonstrate how these design principles apply to the accessibility of the MEGAphone concept.
        \end{enumerate} 
    \item How can the 'Right to Repair' mantra be used in support of the universally designed MEGAphone case?
        \begin{enumerate}
        \item[-] Demonstrate how the MEGAphone concept can be made easier to repair and maintain.
        \end{enumerate} 
    \item What are the effects of COVID-19 on Digital Sovereignty, Universal Design and the MEGAphone project? %%this perhaps too late in project to mention? need to review
        \begin{enumerate}
        \item[-] Discuss the effects of not being able to source parts under pandemic restrictions due to extensive outsourcing to other countries.
        \item[-] Discuss the importance of end user input in the design of the MEGAphone concept and how COVID-19 affects that ability.
        \end{enumerate} 
\end{enumerate}

%----------------------------------------------------------------------------------------
%	SECTION 5
%----------------------------------------------------------------------------------------
\section{Scope of Project}

This project will look at the history and importance of UD and DS (specifically the 'Right to Repair') in the design of products today as well as the relationship between those two concepts.
The development of a unique computer-aided universally designed chassis prototype graded against the seven design principles \cite{sevenprinciples} will be presented, along with multiple PCB designs that support this chassis, the existing MEGAphone PCB and hardware-implimented accessibility 'software' written in VHDL for the MEGA65 operating system.
%% possible update, revise this at some point
%% what is out of scope? software out of scope etc

% %----------------------------------------------------------------------------------------
% %	SECTION 6
% %----------------------------------------------------------------------------------------
\section{Contribution to the Field} %% past tense!

The MEGAphone device is a platform that exists to provide people with true control over their data, and by extension their right to repair, modify and maintain their device.
This project aims to support the MEGAphone project by developing an interface that enhances the accessibility of the device.
It is the belief of the author that a unique project such as the MEGAphone, with the ability to give users a secure and truely open telephony platform, should be accessible to everyone regardless of their age or ability.
This thesis investigated exactly how that could be achieved in order to deliver a MEGAphone device chassis, supporting existing PCBs and hardware-implemented user interface that embodies the vision of UD and DS.\vspace{5mm} %%go through an abbreviate digital sov, also revise
%%% MAYBE ADD THIS TO CONCLUDING CHAPTER AS WELL
%% maybe numbered list, key contributions are...
% only define abbr once
%%table captions above, figure captions below
\newline
Specifically, the author's key contributions to the field are:
\begin{enumerate}
    \item An examination of the relationship between Digital Sovereignty and Universal Design.
    \item Analysis of the existing MEGAphone PCB design from the perspective of UD.
    \item The creation through iteration of a UD housing for the MEGAphone PCB.
    \item The improvement of the MEGAphone device layout to improve its UD properties.
    \item The design of practical modifications to the MEGAphone PCB design that allow the majority of the UD improvements to be realised, without requiring a complete and costly redesign of the MEGAphone PCB.
    \item The proposal of further improvements to the MEGAphone PCB and case designs to further improve their UD properties.
    \item The creation of a MEGAphone device prototype, including UD features and housing, that can be used in future UD studies of the MEGAphone.
\end{enumerate}

%----------------------------------------------------------------------------------------
%	SECTION 7
%----------------------------------------------------------------------------------------
\section{Structure of Thesis}

Chapter 1 addresses the aims of the project, provides some context on the MEGAphone and lists the questions that this thesis aims to answer.
Chapter 2 investigates how Universal Design can help to overcome many issues in product design that people with disabilities face without the need for specialized equipment. 
Additionally, the lack of Digital Sovereignty in electronics devices is addressed as well as why it is imperative that it be preserved.
Chapter 3 focuses on the methods and tools used to plan, design and manufacture this project.
Chapter 4 discusses the design process and the incremental changes made to the design toward a definitive prototype, supplemented by a discussion in regard to the outcome of the project. 
Chapter 5 provides concluding remarks as well as a final evaluation of the case prototype, PCBs, hardware-implemented 'software' features and future work to be undertaken.
%%% ADDRESS THIS SECTION AGAIN AT THE END OF WRITING!!